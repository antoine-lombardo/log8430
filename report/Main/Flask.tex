\section{Flask Application Deployment Procedure} \label{T1}

\paragraph{}The deployment of our application can be done using the first option of our run.sh script. This script guides the user while configuring AWS, then run a Python script that uses Boto3 for creating Security Groups, Instances, Target Groups and Load Balancer. Boto 3 is a AWS SDK that can be used to create, configure, and manage AWS services. That script will automatically flush old instances and other configurations then configure everything correctly to make sure everything always work as expected.
\paragraph{}The deployment of the Flask application in every instances is done with the use of a user\_data script, which allows us to perform automated configuration tasks and run the Flask app instance starts. Our user\_data script first installs git on each instance, then clone our git repository. It then install all the Python requirements then run the Flaks app on the instance. After that, each instance starts to respond to the port 80.
\paragraph{}To make sure that our Flask app keeps running, we used nohup, which instructs the system to continue running it even if the session is disconnected. That way, we can be sure that the Flask app we started won't stop.